\documentclass[12pt]{article}
\usepackage{geometry}
\usepackage{amsmath}
\usepackage{amssymb}
\usepackage{enumitem}
\usepackage{fancyhdr}
\usepackage{tikz}
\usepackage[utf8]{inputenc}
\usepackage[spanish]{babel}
\usepackage{lmodern}
\usepackage[T1]{fontenc}
\usepackage{amsthm}
\usepackage{amsfonts}
\usepackage{graphicx}
\usepackage{empheq}
\usepackage{pifont}
\usepackage{stmaryrd}
\usepackage{marvosym}
\usepackage{qtree}
\usepackage{makeidx}
\usepackage{hyperref}
\usepackage{setspace}
\usepackage{bbm}
%\usepackage{flexisym}
\usepackage{amsmath}
\usepackage{geometry}
 \geometry{
 a4paper,
 total={170mm,257mm},
 left=15mm,
 right=15mm,
 top=20mm,
 bottom=20mm
 }

\usepackage[linesnumbered,ruled,vlined,spanish,onelanguage]{algorithm2e}

\theoremstyle{plain}

\theoremstyle{definition}
\newtheorem*{theorem*}{Teorema}
\theoremstyle{definition}
\newtheorem{theorem}{Teorema}
\theoremstyle{definition}
\newtheorem*{solution}{Solución}

\usepackage{amssymb, enumerate}
\usepackage{amscd, textcomp}




\usetikzlibrary{trees}
\pagestyle{fancy}


\usepackage{lmodern}


% Command "alignedbox{}{}" for a box within an align environment
% Source: http://www.latex-community.org/forum/viewtopic.php?f=46&t=8144
\newlength\dlf  % Define a new measure, dlf
\newcommand\alignedbox[2]{
% Argument #1 = before & if there were no box (lhs)
% Argument #2 = after & if there were no box (rhs)
&  % Alignment sign of the line
{
\settowidth\dlf{$\displaystyle #1$}  
    % The width of \dlf is the width of the lhs, with a displaystyle font
\addtolength\dlf{\fboxsep+\fboxrule}  
    % Add to it the distance to the box, and the width of the line of the box
\hspace{-\dlf}  
    % Move everything dlf units to the left, so t
\boxed{#1 #2}
    % box around lhs and rhs
}
}


\lhead{Juan Sebastián Corredor %Rodriguez
- jucorredorr@unal.edu.co}
\chead{}
\rhead{Homework No.1 - Neural Networks}

\DeclareMathOperator{\End}{End}
\DeclareMathOperator{\Hom}{\operatorname{Hom}}
\DeclareMathOperator{\Der}{\operatorname{Der}}
\DeclareMathOperator{\GL}{\operatorname{GL}}
\DeclareMathOperator{\SL}{\operatorname{SL}}
\DeclareMathOperator{\SO}{\operatorname{SO}}
\DeclareMathOperator{\Ort}{\operatorname{O}}
\newcommand{\ca}{\mathtt{c}}
\DeclareMathOperator{\Tr}{\operatorname{Tr}}
\newcommand{\id}{\mathrm{I}}
\DeclareMathOperator{\ad}{\mathtt{ad}}
\newcommand{\Id}{\mathrm{Id}}
\newcommand{\pr}{\mathtt{pr}}
\newcommand\dual[1]{{#1}^{\vee}}
\newcommand{\trace}{tr}
\newcommand{\Ker}{\text{Ker}}
\newcommand{\p}{\text{.}}
\newcommand{\prob}{\text{Pr}}
\newcommand{\re}{\text{rep}}
\newcommand{\var}{\text{Var}}
\newcommand{\ra}{R_C}
\newcommand{\F}{\mathbb{F}}
\newcommand{\R}{\mathbb{R}}
\newcommand{\Z}{\mathbb{Z}}
\newcommand{\N}{\mathbb{N}}
\newcommand{\Order}{\mathcal{O}}
\newcommand{\C}{\mathcal{C}}
\DeclareMathOperator{\rank}{rank}
\DeclareMathOperator{\tr}{tr}
\DeclareMathOperator{\Var}{Var}
\DeclareMathOperator{\cov}{Cov}

\newenvironment{miscases}
  {\left.\begin{aligned}}
  {\end{aligned}\right\rbrace}

\begin{document}

\section*{A Brief Application of NNs}
\noindent The main idea of this document is to explain the goal and results of an article that uses NNs to solve a problem. In particular, I picked an article about the application of CNNs  (Convolutional Neural Networks) for classification of YouTube videos: \cite{6909619}.     \subsection*{Why did I pick this article?}
\noindent Until now, my favorite NNs are the CNNs because of its architecture and intuition, which for me is to understand an observation in a local way (from the mathematical point of view) by evaluating groups of neighbors. So, I landed in \cite{6909619}, a highly cited article (with more than $3400$ citations), by searching in Google Scholar.   
\subsection*{What about the article?}
\noindent The name of the article is \textit{Large-scale video classification with convolutional neural networks} and the authors are George Toderice, Sanketh Shetty, Thomas Leung, Rahul Sukthankar from Google, Li Fei-Fei from Stanford University and Andres Karpathy from both Google and Stanford University.\\
\\
All the information about the article, such as git repositories and datasets, can be found in this link:
\begin{center}
\url{https://www.cv-foundation.org/openaccess/content_cvpr_2014/html/Karpathy_Large-scale_Video_Classification_2014_CVPR_paper.html}    
\end{center}
\subsection*{What is the problem to be solved?}
\noindent 
\subsection*{Application of NNs to this problem}
\noindent
\subsection*{Results and conclusions of the article}
\noindent

%Code for doing problem-solution points
%\noindent \textbf{Problem 1.} 
%\begin{solution}
%\end{solution}
%\begin{flushright}
%$\blacksquare$
%\end{flushright}

\bibliographystyle{apalike}
\bibliography{References.bib}
   
\end{document}
