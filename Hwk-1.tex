\documentclass[12pt]{article}
\usepackage{geometry}
\usepackage{amsmath}
\usepackage{amssymb}
\usepackage{enumitem}
\usepackage{fancyhdr}
\usepackage{tikz}
\usepackage[utf8]{inputenc}
\usepackage[spanish]{babel}
\usepackage{lmodern}
\usepackage[T1]{fontenc}
\usepackage{amsthm}
\usepackage{amsfonts}
\usepackage{graphicx}
\usepackage{empheq}
\usepackage{pifont}
\usepackage{stmaryrd}
\usepackage{marvosym}
\usepackage{qtree}
\usepackage{makeidx}
\usepackage{hyperref}
\usepackage{setspace}
\usepackage{bbm}
%\usepackage{flexisym}
\usepackage{amsmath}
\usepackage{geometry}
 \geometry{
 a4paper,
 total={170mm,257mm},
 left=15mm,
 right=15mm,
 top=20mm,
 bottom=20mm
 }

\usepackage[linesnumbered,ruled,vlined,spanish,onelanguage]{algorithm2e}

\theoremstyle{plain}

\theoremstyle{definition}
\newtheorem*{theorem*}{Teorema}
\theoremstyle{definition}
\newtheorem{theorem}{Teorema}
\theoremstyle{definition}
\newtheorem*{solution}{Solución}

\usepackage{amssymb, enumerate}
\usepackage{amscd, textcomp}




\usetikzlibrary{trees}
\pagestyle{fancy}


\usepackage{lmodern}


% Command "alignedbox{}{}" for a box within an align environment
% Source: http://www.latex-community.org/forum/viewtopic.php?f=46&t=8144
\newlength\dlf  % Define a new measure, dlf
\newcommand\alignedbox[2]{
% Argument #1 = before & if there were no box (lhs)
% Argument #2 = after & if there were no box (rhs)
&  % Alignment sign of the line
{
\settowidth\dlf{$\displaystyle #1$}  
    % The width of \dlf is the width of the lhs, with a displaystyle font
\addtolength\dlf{\fboxsep+\fboxrule}  
    % Add to it the distance to the box, and the width of the line of the box
\hspace{-\dlf}  
    % Move everything dlf units to the left, so t
\boxed{#1 #2}
    % box around lhs and rhs
}
}


\lhead{\href{https://github.com/juanse1608}{Juan Sebastián Corredor} %Rodriguez
- jucorredorr@unal.edu.co}
\chead{}
\rhead{Homework No.1 - Neural Networks}

\DeclareMathOperator{\End}{End}
\DeclareMathOperator{\Hom}{\operatorname{Hom}}
\DeclareMathOperator{\Der}{\operatorname{Der}}
\DeclareMathOperator{\GL}{\operatorname{GL}}
\DeclareMathOperator{\SL}{\operatorname{SL}}
\DeclareMathOperator{\SO}{\operatorname{SO}}
\DeclareMathOperator{\Ort}{\operatorname{O}}
\newcommand{\ca}{\mathtt{c}}
\DeclareMathOperator{\Tr}{\operatorname{Tr}}
\newcommand{\id}{\mathrm{I}}
\DeclareMathOperator{\ad}{\mathtt{ad}}
\newcommand{\Id}{\mathrm{Id}}
\newcommand{\pr}{\mathtt{pr}}
\newcommand\dual[1]{{#1}^{\vee}}
\newcommand{\trace}{tr}
\newcommand{\Ker}{\text{Ker}}
\newcommand{\p}{\text{.}}
\newcommand{\prob}{\text{Pr}}
\newcommand{\re}{\text{rep}}
\newcommand{\var}{\text{Var}}
\newcommand{\ra}{R_C}
\newcommand{\F}{\mathbb{F}}
\newcommand{\R}{\mathbb{R}}
\newcommand{\Z}{\mathbb{Z}}
\newcommand{\N}{\mathbb{N}}
\newcommand{\Order}{\mathcal{O}}
\newcommand{\C}{\mathcal{C}}
\DeclareMathOperator{\rank}{rank}
\DeclareMathOperator{\tr}{tr}
\DeclareMathOperator{\Var}{Var}
\DeclareMathOperator{\cov}{Cov}

\newenvironment{miscases}
  {\left.\begin{aligned}}
  {\end{aligned}\right\rbrace}

\begin{document}

\section*{A Brief Application of NNs}
\noindent The main idea of this document is to explain the goal and results of an article that uses NNs to solve a problem. In particular, I picked an article about the application of CNNs  (Convolutional Neural Networks) for classification of YouTube videos: \cite{6909619}.     \subsection*{Why did I pick this article?}
\noindent Until now, my favorite NNs are the CNNs because of its architecture and intuition, which for me is to understand an observation in a local way (from the mathematical point of view) by evaluating groups of neighbors. So, I landed in \cite{6909619}, a highly cited article (with more than $3400$ citations), by searching in Google Scholar.   
\subsection*{What about the article?}
\noindent The name of the article is \textit{Large-scale video classification with convolutional neural networks} and the authors are George Toderice, Sanketh Shetty, Thomas Leung, Rahul Sukthankar from Google, Li Fei-Fei from Stanford University and Andres Karpathy from both Google and Stanford University.\\
\\
All the information about the article, such as git repositories and datasets, can be found in this link:
\begin{center}
\url{https://www.cv-foundation.org/openaccess/content_cvpr_2014/html/Karpathy_Large-scale_Video_Classification_2014_CVPR_paper.html}    
\end{center}
\subsection*{What is the problem to be solved?}
\noindent The main goal is to classify videos using CNNs architectures. More precisely, the dataset contains about a million sports videos obtained from YouTube belonging to $487$ classes.\\
\\ \cite{6909619} wanted to see the performance of CNNs in this kind of data where an observation it's not only a group of frames, but a temporal evolution of its.It is important to mention that, back then, there were no so much work done for large-scale videos, according to \cite{6909619}.\\
\\
On the other hand, the authors also want to know if the features learned by the networks can be generalized to other datasets i.e. the so called \textit{Transfer Learning}.
\subsection*{Application of NNs to this problem}
\noindent \cite{6909619} establishes that temporal variability of videos makes it difficult to be processed with a fixed-sized architecture. In that particular way, it is proposed to extend the connectivity of the CNN in
time dimension to learn in a spatio-temporal way. To do this, \cite{6909619} propose to use the following architectures:
\begin{itemize}
    \item[1.] Single Frame: A usual single frame CNN architecture. 
    \item[2.] Early Fusion: Combines information across a all the time window. This allows the CNN to detect local motion, direction and speed.
    \item[3.] Late Fusion: Places two separate single framed in order to evaluate global motion in the video. \item[4.] Slow Fusion: Combination between early and late fusion. 
\end{itemize}
As a matter of computational time, \cite{6909619} proposes a Multiresolution Architecture to mitigate the computational cost, by processing two separate streams:
\begin{itemize}
    \item[1.] Context Stream: Learns with low-resolution frames.
    \item[2.] Fovea Stream: Learns with high-resolution frames  on the middle portion of the frame.
\end{itemize}
In a brief way, this was the use of NNs to solve the problems mentioned above. 
\subsection*{Results of the article}
\noindent \cite{6909619} obtained the following main results:
\begin{itemize}
    \item[1.] Multiresolution Single Frame CNN Accuracy: $60.0\%$.
    \item[2.] Early Fusion CNN Accuracy: $57.7\%$.
    \item[3.] Late Fusion CNN Accuracy: $59.3\%$.
    \item[4.] Slow Fusion CNN Accuracy: $60.9\%$.
    \item[5.] CNN Average (Multi + Early + Late + Slow): $63.9\%$.
    \item[6.] Feature Based Approach: $55.3\%$.
\end{itemize}
Note that the average of all CNNs used achieves a significant improvement ($55.3\%$ to $63.9\%$) with respect to the feature based approach. On the other hand, single frame CNN have almost the same accuracy that the better the fusion method (slow fusion).\\
\\ As a matter of fact, in terms of \textit{transfer learning}, works better, after a process of fine tuning (as minimum) the top $3$ layers, that feature based approach ($59.0\%$ to $65.4\%$) on another known sports dataset: UCF-101.

%Code for doing problem-solution points
%\noindent \textbf{Problem 1.} 
%\begin{solution}
%\end{solution}
%\begin{flushright}
%$\blacksquare$
%\end{flushright}

%References
\newpage
\bibliographystyle{apalike}
\bibliography{References.bib}
\end{document}
